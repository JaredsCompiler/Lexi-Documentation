\documentclass{article}

\usepackage{fancyvrb}
\usepackage [english]{babel}
\usepackage [autostyle, english = american]{csquotes}
\usepackage{graphicx}
\usepackage{hyperref}

\MakeOuterQuote{"}
\graphicspath{ {./assets} }

\title{Lexi | Lexical Analyzer}
\author{Jared Dyreson, Chris Nutter}
\date{California State University, Fullerton}

\begin{document}
\maketitle
\tableofcontents
\newpage

\section{Introduction}
This project was intended to be a lexical analyzer for our compiler and was named "Lexi".
The goal of Lexi is to parse out the contents of a source document and generate meaningful lexemes.
These lexemes were to adhere to a specific set of regular expressions to define a token.

\section{How to Use}

All code is compiled under the latest version of "clang" and "g++".

\begin{enumerate}
		\item Use on a Unix-based operating system, preferably Linux
		\item Clone the repository from \href{https://github.com/JaredsCompiler/Lexi}{\underline{here}}
		\item Enter the directory named "Lexi"
		\item Run 'make test' to conduct all the tests without having to present command line arguments
\end{enumerate}

\section{Design | Regular Expression}
The way Lexi parses each line and determines the identifier type is through the use of regular expressions. 
Being able to determine the identifier is crucial in defining the token's contents.
Lexi after processing the file and creating a vector of strings that parses line by line which is then fed through a function that reads each character and determines one of the each lexeme types.

\subsection{Comments}

These are any sequence of characters enclosed by two exclamation marks, and can be embedded in a line with other code.
Multiple comments in a line are supported.

\begin{figure}[!htpb]
\centering
\begin{Verbatim}[frame=single]
(!.*!)
\end{Verbatim}
\end{figure}

\newpage


\subsection{Identifier}

These are any sequence of characters and numbers starting with an alphanumeric.
It should also be known that we must compare any identifier matched to a list of reserved words.
These reserved words are not to be used as variable names as the are used for data types, control-flow operators, and other key-defining words for the language.
This is only supported in this iteration because \emph{Flex} and \emph{Bison} do this quite well.

\begin{figure}[!htpb]
\centering
\begin{Verbatim}[frame=single]
([a-zA-Z]+(\d*)?)
\end{Verbatim}
\end{figure}

\subsection{Numbers}

These are any integer like number, such as floating point and ordinary base 10 integers.
Lexi also supports signed/unsigned numbers.

\begin{figure}[!htpb]
\centering
\begin{Verbatim}[frame=single]
(?:\b)([-+]?\d*.?\d+)?(?=\b)
\end{Verbatim}
\end{figure}


\subsection{Operators}

\begin{flushleft}

These operators should be broken down into two distinct categories:
\begin{itemize}
\item Unary
\item Binary
\end{itemize}

Where the unary operator only takes in one argument and returns a value.
The binary operator takes in two arguments and returns one value.
\end{flushleft}

\subsubsection{Binary | Comparators}
\begin{flushleft}
\begin{figure}[!htpb]
\centering
\begin{Verbatim}[frame=single]
(>|<|>=|<=|&&|||)
\end{Verbatim}
\end{figure}

The operators supported here are as follows:
\begin{itemize}
\item Less than
\item Less than or equal to
\item Greater than
\item Greater than or equal to
\item Logical AND, OR
\end{itemize}
\end{flushleft}

\subsubsection{Binary | Operands}
\begin{flushleft}
\begin{figure}[!htpb]
\centering
\begin{Verbatim}[frame=single]
(+|-|*|/|=|>|<|>=|<=|&|||%|^)
\end{Verbatim}
\end{figure}

\begin{flushleft}
The operators supported here are as follows:
\begin{itemize}
\item Addition
\item Subtraction
\item Multiplication
\item Division
\item Modulo
\item Bitwise AND, OR, XOR
\end{itemize}

There is a subset of these binary operators that are exclusively used as comparators and deserve to be in their own separate category.

\end{flushleft}


\subsubsection{Unary}
\begin{flushleft}
\begin{figure}[!htpb]
\centering
\begin{Verbatim}[frame=single]
(^!$)
\end{Verbatim}
\end{figure}

The only operator here is the invert/not operator, which will take in boolean value and return the inverse of it.
For example, if the value is set to true and you apply the not operator, you will get false.
\end{flushleft}

\newpage


\newpage

\subsection{Separators}

These symbols will typically enclose an expression that needs either be evaluated, contains string literals or separate multiple values of the same type.
Here, it is easier to note the LHS (left hand side) and RHS (right hand side) variants of each separator, as they are important to distinguish in the beginning.

\subsubsection{Brackets}

\begin{figure}[!htpb]
\centering
\begin{Verbatim}[frame=single]
({)
\end{Verbatim}
\caption{LHS Bracket}
\end{figure}


\begin{figure}[!htpb]
\centering
\begin{Verbatim}[frame=single]
(})
\end{Verbatim}
\caption{RHS Bracket}
\end{figure}

\subsubsection{Comma}
\begin{figure}[!htpb]
\centering
\begin{Verbatim}[frame=single]
(\,)
\end{Verbatim}
\caption{Comma Character}
\end{figure}

Identifiers are comma separated for things like instantiation for multiple variables of the same time.

\newpage

\subsubsection{Parenthesis}
\begin{figure}[!htpb]
\centering
\begin{Verbatim}[frame=single]
(()
\end{Verbatim}
\caption{LHS Parenthesis}
\end{figure}


\begin{figure}[!htpb]
\centering
\begin{Verbatim}[frame=single]
())
\end{Verbatim}
\caption{RHS Parenthesis}
\end{figure}


\subsubsection{String Literals}
\begin{figure}[!htpb]
\centering
\begin{Verbatim}[frame=single]
(\"|\')
\end{Verbatim}
\end{figure}

These are expressions should contain alphanumeric characters enclosed by either single or double quotes.
Single quotes should be used for single char like data member.
Double quotes are string literals.

\subsection{Terminators}

These symbols signal the end of an expression.
\begin{figure}[!htpb]
\centering
\begin{Verbatim}[frame=single]
(\;|\$)
\end{Verbatim}
\end{figure}

\end{flushleft}
\end{document}
